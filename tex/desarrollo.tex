\vspace{5px}
\subsection{Tres Colores} 

Dentro de este filtro elegimos tomar de a 4 píxeles donde dividimos en 2 partes el cálculo de estos mismos. 
La primera parte consiste en calcular $W$, para cada uno de los píxeles. Para esto duplicamos los bytes de colores individuales en 4 registros distintos para luego sumarlos y obtengamos algo como:

\begin{table}[h]
\begin{center}
\begin{tabular}{|c|c|c|c|c|}
\hline
      & dword     & dword     & dword     & dword     \\ \hline
xmm1: & R     & R     & R     & R     \\ \hline
+     & +     & +     & +     & +     \\ \hline
xmm2: & G     & G     & G     & G     \\ \hline
+     & +     & +     & +     & +     \\ \hline
xmm3: & B     & B     & B     & B     \\ \hline
=     & =     & =     & =     & =     \\ \hline
xmm1: & R+G+B & R+G+B & R+G+B & R+G+B \\ \hline
\end{tabular}
\end{center}
\end{table}

De esta manera obtenemos el cálculo $R+G+B$ de 4 píxeles. Una vez obtenida la suma anterior, necesitamos dividirla por 3 para obtener el $W$ deseado. Para evitar la pérdida de precisión en este pasó, convertimos estos valores a $single$ $precision$ $floats$. Una vez realizada esta conversión y la división volvemos a formatear a $int$:

\begin{table}[h]
\begin{center}
\begin{tabular}{|c|c|c|c|c|}
\hline
      & dword     & dword     & dword     & dword     \\ \hline
xmm1: & W     & W     & W     & W     \\ \hline
\end{tabular}
\end{center}
\end{table}

Para elegir entre los colores: $crema, rojo, verde$ en cada pixel se realiza una serie de comparaciones para evaluar el valor correcto. Para calcular estos valores se pasa por comparaciones de los 3 colores predefinidos ($crema$, $rojo$ y $verdes$). Estos mismos valores se encuentran cargados en la sección de .data ya multiplicados por 3, ya que el valor solo se utiliza así. Por lo tanto para ahorrar instrucciones se carga en memoria de antemano. 

Se decidió realizar primero las sumas de cada píxel antes de realizar la división por 4 para evitar perder precisión. Esto mismo nos genera el problema de posibles overflow en el único byte con el que veníamos trabajando, ya que cada byte de los colores predefinidos es multiplicado por 3 y eso de por si puede generar overflow. Por lo que se decidió realizar los siguientes pasos para el procesamiento de cada color:

\begin{itemize}
  \item Dividimos el proceso de los 4 píxeles en dos, en pos de evitar overflow por lo primero tomaremos los dos primeros píxeles extendidos. Realizamos esta división de procesamiento para evitar overflow, ya que al multiplicar por 3 las constantes que tenemos predefinidas algunas superan los 2 bytes disponibles que tenemos, y si en adición sumamos W esto solo empeora la situación. Cabe aclarar que la división por 4 para cada píxel la realizamos luego de sumar ambos términos para no perder precisión.
  
  \item Miremos especialmente una iteracion para uno de los colores predefinidos: A los dos píxeles primero agregamos los correspondientes valores de $W$, los dividimos por 4 y al finalizar realizamos als comparaciones para elegir cuales de estos colores deberían llegar el color predefinido actual. Realizamos el mismo procedimiento para los segundos píxeles. 
  Este mismo proceso se realiza una vez para cada uno de los colores predefinidos $crema$, $rojo$ y $verdes$.
\end{itemize}

Una vez realizado este procesamiento obtenemos el siguiente registro:


\begin{table}[h]
\begin{center}
\begin{tabular}{|c|c|c|c|c|}
\hline
      & dword     & dword     & dword     & dword     \\ \hline
xmm1: & (CVR + W)/4     & (CVR + W)/4     & (CVR + W)/4     & (CVR + W)/4     \\ \hline
\end{tabular}
\end{center}
\end{table}

Siendo CVR el color correspondiente para cada pixel.
Una vez obtenido este resultado escribimos en memoria el dato, para luego continuar ciclando la imagen. 


\subsection{Efecto Bayer}

Sobre este filtro dividimos las iteraciones en 2 partes una para cada fila:

\begin{itemize}
  \item En la primera tomaremos el patrón $Azul$ y $Verde$
  \item Y en la segunda tendremos el patrón $Verde$ y $Rojo$
\end{itemize}

Cabe destacar que tomamos en orden inverso los patrones a diferencia de los que se muestran en la imagen de ejemplo ya que se comienza desde el borde inferior derecho.

En el ciclo principal del filtro tomamos de a 2 filas:


\begin{table}[h]
\begin{center}
\begin{tabular}{|c|c|c|c|c|}
\hline
 Rojo   & Verde &  Rojo   & Verde & ...    \\ \hline
 Verde  & Azul & Verde  & Azul & ...   \\ \hline
\end{tabular}
\end{center}
\end{table}

Dentro de estos mismos, aplicamos un ciclo específico para cada una de las filas. Tomando por ejemplo la primera:

\vspace{6px}
\begin{center}
\includegraphics[width=10.5cm, height=1.5
cm]{images/efectobayer_unafila.png}
\end{center}
\vspace{6px}

Cuando tomamos esta fila comenzaremos a iterar de a 4 píxeles uno trabajando simultáneamente para ir filtrando los colores de estos píxeles utilizando la instrucción $shuffle$ para eliminar los demás colores. De esta manera completamos toda la fila.
Una vez completada esta fila se procede de manera similar con la siguiente fila 


\subsection{Cambia Color}

Para comenzar este filtro primero calculamos los valores de $N_r$, $N_b$, $N_b$, $C_r$, $C_g$, $C_b$, $lim$ a partir de los parámetros estableciendo registros $XMM$ llenos de estos mismos valores para así poder trabajar de a 4 píxeles a la vez. 

Una vez guardados los valores mencionados comenzaremos a ciclar sobre la imagen trayendo de a 4 píxeles. Paso siguiente, se calculo paso a paso el valor de $d$ para cada uno de estos píxeles con los siguientes pasos:

%Acá voy a agregar alguna manera bonita de poner los xmm, la estoy pensando todavía
\begin{itemize}
  \item Calcular los $\Delta B$, $\Delta R$, $\Delta G$ para cada píxel, por lo que obtendremos algo similar a $xmm = | \Delta R | \Delta R | \Delta R | \Delta R |$ para cada color. Este proceso se realiza en $int$ hasta la unión de todas las sumas que contienen ya que nunca superamos 1 byte en cada $dword$ que manejamos. Antes de realizar los últimos pasos para calcular $d$ y $c$, convertimos a float para no perder precision en la división por 256 que se encuentra adento, y tampoco para saturar ya que la suma de todos los valores antes de aplicar la raíz cuadrada podría saturar. Cabe destacar que también se pasa a float para calcular correctamente la raíz cuadrada. 
  \item El siguiente paso es calcular el valor designado en caso de que $d<lim$. Este mismo se calcula con 4 píxeles también y para cada uno realizamos las operaciones para generar $1-c$ y $N_r,N_b,N_g$ de cada byte del píxel, antes de multiplicar $1-c$ por cada respectivo valor, pasamos a float para no perder precisión. 
\end{itemize}

Una vez terminados estos pasos tenemos el resultado procesado de 4 píxeles que volvemos a dejar en memoria.


\subsection{Edge Sobel}
El algoritmo comenzara llenando la parte que se procesa (toda la matriz, excepto los bordes),
y luego completara los bordes de la matriz con 0.
Tiene cuatro diferentes tipos de iteraciones y un caso que es tratado de forma diferente.
\begin{itemize}
\item Llenado de la matriz con los pixeles procesados, se excluyen los bordes pues no se pueden aplicar los operadores.
\item Llenado de la última fila a procesar.
\item Llenado de los bordes de los costados.
\item El ultimo caso es para poder llenar la primer fila y la ultima con ceros, la forma de iterar es la misma pero cambia el lugar desde donde inicia cada una.
\end{itemize}
\subsubsection{Llenado de los pixeles procesados usando parte baja de un xmm} Se tomara la parte baja de un registro xmm y los 8 bytes de la parte baja serán extendidos a 8 words para que al hacer las cuentas no perdamos precisión, como se calcularán 4 pixeles, estaremos trabajando con 6 bytes, los 10 bytes restantes no se utilizarán. Esto solo sirve para llenar parte de la matriz, por que al querer seguir iterando, llegaremos a la última posición y necesitariamos leer 10 bytes demás, pero nos saldriamos de la matriz.

\subsubsection{Llenado de los pixeles procesadosusando parte alta de un xmm} Responde al problema que surgio anteriormente, entonces ahora al leer en un registro xmm tranjaremos con la parte alta, como necesitamos procesar 4 pixeles a la hora de acceder a la matriz necesitaremos estar 10 bytes (posiciones) atras, asi poder acceder a los 6 bytes con los que se procesan 4, luego por la forma en como iniciamos y dado que la cantidad de pixeles en la fila es muliplo de 8, no podremos  procesar los dos ultimos bytes y estos se harán como caso aparte.




\subsubsection{Llenado de los bordes de los costados} La matriz es accedienda desde último byte de la primer fila, copia su valor en un xmm, hace un shift de 2 bytes para poner los 2 primeros bytes en 0 y luego un shift de 2 bytes para reacomodar, con esto se puso en 0 los 2 byte de los costados, luego se incementa en cantidad de columnas a la posición y se procede de la misma forma hasta llegar a la ante-ultima fila.

\subsubsection{Llenado de los bordes superior e inferior}
	Para la \textbf{primer fila}, primero se lee con el registro xmm y se shiftean 8 bytes, se divide la cantidad de columnas por 8 y se procede a iterar cantidad de (cantidad de pixeles en fila/8) -1 para poder llenas la primer fila.
	Para la \textbf{ultima fila}, se lee desde 8 bytes antes y se hace un shift para poder poner en 0 los primeros 8 bytes de la ultima fila, luego se procede a iterar cantidad de (cantidad de pixeles en fila/8) -1 para poder llenar la matriz.

